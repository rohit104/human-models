\section{Experimental results}
\label{sec:experiments}

To prepare data for human subjects to review, we fit three different
topic models on two corpora.  In this section, we describe how we
prepared the corpora, fit the models, and created the tasks described
in \mysec{sec:tasks}.  We then present the results of these human
trials and compare them to metrics traditionally used to evaluate
topic models.

\subsection{Models and corpora}
\label{sec:expermodels}

In this work we study three topic models: probabilistic latent
semantic indexing (pLSI)~\cite{hofmann-99}, latent Dirichlet
allocation (LDA)~\cite{blei-03}, and the correlated topic model
(CTM)~\cite{blei-06}, which are all mixed membership
models~\cite{Erosheva:2004ph}.  The number of latent topics, $K$, is a free
parameter in each of the models; here we explore this with $K=50$,
$100$ and $150$.  The remaining parameters -- $\beta_k$, the topic
multinomial distribution for topic $k$; and $\theta_d$, the topic
mixture proportions for document $d$ -- are inferred from data.  The
three models differ in how these latent parameters are inferred.

% Conditioned on $\theta_d$, the likelihood of the observed bag of words
% for document $d$, $\bm{w}_d$, for all three models exhibits the same
% probabilistic form,
% \begin{align}\label{eq:likelihood}
% p(\bm{w}_d|\theta_d) = \prod_{n=1}^{N}\sum_{z_{n}=1}^K p(w_{n}|z_{n}, \beta_{1:K}) p(z_{n}|\theta).
% \end{align}
% However, the different assumptions about the mixture proportions
% $\theta_d$ used by the three models lead to significantly different
% latent spaces.

\begin{list}{\textbf}{\leftmargin=0pt}
\setlength{\itemindent}{25pt}
\setlength{\itemsep}{0pt}
\item[{\bf pLSI}] In pLSI, the topic mixture proportions $\theta_d$ are a parameter for
% Conditioned on $\theta_d$, the likelihood of the observed bag of words
% for document $d$, $\bm{w}_d$, for all three models exhibits the same
% probabilistic form,
% \begin{align}\label{eq:likelihood}
% p(\bm{w}_d|\theta_d) = \prod_{n=1}^{N}\sum_{z_{n}=1}^K p(w_{n}|z_{n}, \beta_{1:K}) p(z_{n}|\theta).
% \end{align}
% However, the different assumptions about the mixture proportions
% $\theta_d$ used by the three models lead to significantly different
% latent spaces.
  each document.  Thus, pLSI is not a fully generative model, and the number
  of parameters grows linearly with the number of documents.
% jc1: ???
% Moreover due to its non-generative nature, inference needs to be
% done in the ``folding-in'' process~\cite{hofmann-99}.
  We fit pLSI using the EM algorithm~\cite{Dempster:1977} but
  regularize pLSI's estimates of $\theta_d$ using pseudo-count
  smoothing, $\alpha = 1$.  

% Under these settings, the maximum \emph{a
%     posteriori} (MAP) estimates found by pLSI correspond to those
%   found by LDA with a symmetric Dirichlet prior on $\theta_d$ with
%   parameter $\alpha$~\cite{Girolami-03}.

\item[{\bf LDA}] LDA is a fully generative model of documents where
  the mixture proportions $\theta_d$ are treated as a random variable
  drawn from a Dirichlet prior
  distribution. %that is%$\theta \sim \textrm{Dirichlet}(\alpha)$.
% \begin{align}\label{eq:dirichlet}
% p(\theta | \alpha) = \frac{\Gamma(\sum_{k=1}^K\alpha_k)}{\prod_{k=1}^K \Gamma(\alpha_k) } \prod_{k=1}^K\theta_k^{\alpha_k-1},
% \end{align}
% where $\Gamma(\cdot)$ is the Gamma function.  
% Due to its generative nature, inference on unseen documents is
% well-defined.  
  Because the direct computation of the posterior is intractable, we
  employ variational inference~\cite{blei-03} and set the symmetric
  Dirichlet prior parameter, $\alpha$, to 1.

\item[{\bf CTM}] In LDA, the components of $\theta_d$ are nearly independent
  (i.e., $\theta_d$ is statistically neutral).
% This leads to the possibly-unrealistic modeling assumption
% that the presence of one topic in a document is not correlated with
% the presence of another topic in the same document. 
CTM allows for a richer covariance structure between topic proportions
by using a logistic normal prior over the topic mixture proportions
$\theta_d$.  For each topic, $k$, a real $\gamma$ is drawn from a
normal distribution and exponentiated.  This set of $K$ non-negative
numbers are then normalized to yield $\theta_d$.  Here, we train the
CTM using variational inference~\cite{blei-06}.
\end{list}

% Using this formulation, CTM is able to capture the correlations among
% different topics and provides a natural way of visualizing the topics
% using the correlations.


%% {\bf TODO: Chong, it seems not fair not to optimize $\alpha$ for pLSI and LDA but optimize that for CTM?}

We train each model on two corpora.  For each corpus, we apply a part
of speech tagger~\cite{schmid-94} and remove all tokens tagged as
proper nouns (this was for the benefit of the human subjects; success
in early experiments required too much encyclopedic knowledge).  Stop
words~\cite{loper-02} and terms occurring in fewer than five documents
are also removed. The two corpora we use are 1.) a collection of 8447
articles from the \textit{New York Times} from the years 1987 to 2007
with a vocabulary size of 8269 unique types and around one million
tokens and 2.)  a sample of 10000 articles from \textit{Wikipedia}
(http://www.wikipedia.org) with a vocabulary size of 15273 unique
types and three million tokens.
% \item \textbf{IMDB}: a collection of xxx reviews from the
%   \textit{Internet Movie Database}
%   (IMDb)\footnote{http://www.imdb.com/}, with a vocabulary size xxx
%   and around xxx words.

\subsection{Evaluation using conventional objective measures}
\label{sec:evalmetrics}

There are several metrics commonly used to evaluate topic models in
the literature~\cite{wallach-09}.  Many of these metrics are
\emph{predictive} metrics; that is, they capture the model's ability
to predict a \emph{test set} of unseen documents after having learned
its parameters from a \emph{training set}.  In this work, we set aside
20\% of the documents in each corpus as a test set and train on the
remaining 80\% of documents.  We then compute predictive rank and
predictive log likelihood.

To ensure consistency of evaluation across different models, we follow
Teh et al.'s~\cite{TehKurWel2008} approximation of the predictive
likelihood $p(\textbf{w}_d|D_{\textrm{train}})$ using $p({\bm
  w}_d|D_{\textrm{train}}) \approx p({\bm w}_d|\hat{\theta}_d)$, where
$\hat{\theta}_d$ is a point estimate of the posterior topic
proportions for document $d$. For pLSI $\hat{\theta}_d$ is the MAP
estimate; for LDA and CTM $\hat{\theta}_d$ is the mean of the
variational posterior.
With this information, we can ask what words the model believes will
be in the document and compare it with the document's actual
composition.  Given document $\bm w_d$, we first estimate
$\hat{\theta}_d$ and then for every word in the vocabulary, we compute
$p(w|\hat{\theta}_d) = \sum_z p(w|z)p(z|\hat{\theta}_d)$.  Then we
compute the average rank for the terms that actually appeared in document
$\bm w_d$ (we follow the convention that lower rank is better).


%\paragraph{Predictive perplexity}

% The predictive perplexity of a heldout set $D_{heldout}$ given the
% training set $D_{train}$ is defined as
% \begin{align*} \label{eq:perplexity}
%   {\textrm{perplexity}} = \exp\left\{-\frac{\sum_{d\in D_{\textrm{heldout}}}\log
%       p(\textbf{w}_d|D_{\textrm{train}})}{\sum_{d\in D_{heldout}}N_d}\right\}.
% \end{align*}


% \begin{table}
% \caption{Perplexity comparison for different models on three corpora.}
% \label{fig:perplexity}
% %\vskip 0.15in
% \begin{center}
% \begin{small}
% \begin{sc}
% \begin{tabular}{c|c|cc}
%   \hline
% \multicolumn{4}{c}{Per-word perplexity} \\
% \hline
% \#Topics & Model & NYT & Wikipedia \\
% \hline
% \multirow{3}{*}{50}
% & LDA & 1641.82 & 1980.65 \\
% & CTM & 1755.38 & 2517.14 \\
% & pLSI &  &  \\
% \hline
% \multirow{3}{*}{100}
% & LDA & 1542.21 & 1836.51 \\
% & CTM & 1652.53 & 2327.30 \\
% & pLSI &  &  \\
% \hline
% \multirow{3}{*}{150}
% & LDA & 1485.04 & 1755.31 \\
% & CTM &         & 2227.58 \\
% & pLSI &   \\
% \hline
% \end{tabular}
% \end{sc}
% \end{small}
% \end{center}
% \end{table}

%% We need to decide if we want to keep these metrics...
% \paragraph{Word predictions}

% The first metric is predictive perplexity, which is an objective
% metric used extensively for comparing topic models in the
% literature.  Perplexity is a measure of the
% information content of unseen documents that cannot be captured by the
% model.  Lower perplexity means that the model is better able to
% explain the language of test documents.


The average word likelihood and average rank across all documents in
our test set are shown in Table~\ref{tab:word-prediction}.  These
results are consistent with the values reported in the
literature~\cite{blei-03,blei-06}; in most cases CTM performs best,
followed by LDA.

\begin{table*}
  \caption{Two predictive metrics: predictive log likelihood/predictive rank.  Consistent with values reported in the literature, CTM generally performs the best, followed by LDA, then pLSI.  The bold numbers indicate the best performance in each row.}
\label{tab:word-prediction}
%\vskip 0.15in
\centering
\footnotesize
\begin{sc}
\begin{tabular}{c|c|ccc}
%% \hline
%%  & & \multicolumn{3}{c|}{Model} \\
Corpus & Topics & LDA & CTM & pLSI \\
\hline
\multirow{3}{*}{New York Times}
& 50  &  {\bf-7.3214} / {\bf 784.38}  &    -7.3335 / 788.58 & -7.3384 / 796.43\\
& 100 &  -7.2761 / 778.24  & {\bf -7.2647} / {\bf 762.16} & -7.2834 / 785.05 \\
& 150 &  -7.2477 / 777.32  & -7.2467 / {\bf 755.55} & {\bf -7.2382} /  770.36 \\
\hline
\multirow{3}{*}{Wikipedia}
& 50  & {\bf -7.5257} / 961.86 & -7.5332 / {\bf 936.58} & -7.5378 / 975.88 \\
& 100 & -7.4629 / 935.53 & {\bf -7.4385} / {\bf 880.30} & -7.4748 / 951.78 \\
& 150 & -7.4266 / 929.76 & {\bf -7.3872} / {\bf 852.46} & -7.4355 /  945.29 \\
%\hline
\end{tabular}
\end{sc}
\vspace{0.2in}
\end{table*}

\subsection{Analyzing human evaluations}

The tasks described in \mysec{sec:tasks} were offered on Amazon
Mechanical Turk (http://www.mturk.com), which allows workers (our pool
of prospective subjects) to perform small jobs for a fee through a Web
interface.  No specialized training or knowledge is typically expected
of the workers.  Amazon Mechanical Turk has been successfully used in
the past to develop gold-standard data for natural language
processing~\cite{snow-08} and to label images~\cite{imagenet-cvpr09}.
For both the word intrusion and topic intrusion tasks, we presented
each worker with jobs containing ten of the tasks described in
\mysec{sec:tasks}.  Each job was performed by 8 separate workers, and
workers were paid between \$0.07 -- \$0.15 per job.

\paragraph{Word intrusion}
As described in \mysec{sec:wordintrusion}, the word intrusion task
measures how well the inferred topics match human concepts (using
\emph{model precision}, i.e., how well the intruders detected by the
subjects correspond to those injected into ones found by the topic model).  

Let $\omega^{m}_{k}$ be the index of the intruding word among the words generated
from the $k^{th}$ topic inferred by model $m$.  Further let $i^m_{k, s}$ be
the intruder selected by subject $s$ on the set of words generated from the
$k$th topic inferred by model $m$ and let $S$ denote the number of subjects.  We
define model precision by the fraction of subjects agreeing with the model,
\begin{equation}
  \mathrm{MP}^m_k = \textstyle \sum_{s} \mathds{1}(i^m_{k,s} = \omega^{m}_{k}) / S.
  \label{eq:mp}
\end{equation}


\myfig{fig:precision} shows boxplots of the precision for the three
models on the two corpora.  In most cases LDA performs best. Although
CTM gives better predictive results on held-out likelihood, it does
not perform as well on human evaluations. This may be because CTM
finds correlations between topics and correlations within topics are
confounding factors; the intruder for one topic might be selected from
another highly correlated topic.  The performance of pLSI degrades
with larger numbers of topics, suggesting that
overfitting~\cite{blei-03} might affect interpretability as well as
predictive power.

\myfig{fig:topic_precision} (left) shows examples of topics with high
and low model precisions from the NY Times data fit with LDA using 50
topics. In the example with high precision, the topic words all
coherently express a painting theme.  For the low precision example,
 ``taxis'' did not fit in with the other
political words in the topic, as $87.5\%$ of subjects chose ``taxis''
as the intruder.


The relationship between model precision, $\mathrm{MP}^m_k$, and the
model's estimate of the likelihood of the intruding word in
\myfig{fig:prec_vs_lhood} (top row) is surprising.  The highest
probability did not have the best interpretability; in fact, the trend
was the opposite.  This suggests that as topics become more
fine-grained in models with larger number of topics, they are less
useful for humans.  The downward sloping trend lines in
\myfig{fig:prec_vs_lhood} implying that the models are often trading
improved likelihood for lower interpretability.


The model precision showed a negative correlation (Spearman's $\rho =
-0.235$ averaged across all models, corpora, and topics) with the
number of senses in WordNet of the words displayed to the
subjects~\cite{Miller90} and a slight positive correlation ($\rho =
0.109$) with the average pairwise Jiang-Conrath similarity of
words\footnote{Words without entries in WordNet were ignored; polysemy
  was handled by taking the maximum over all senses of words.  To
  handle words in the same synset (e.g. ``fight'' and ``battle''), the
  similarity function was capped at 10.0.}~\cite{jiang-97}.  

\paragraph{Topic intrusion}

In \mysec{sec:topicintrusion}, we introduced the topic intrusion task
to measure how well a topic model assigns topics to documents.  We
define the \emph{topic log odds} as a quantitative measure of the
agreement between the model and human judgments on this task.  Let
$\hat{\theta}^m_d$ denote model $m$'s point estimate of the topic
proportions vector associated with document $d$ (as described in
\mysec{sec:evalmetrics}).  Further, let $j^m_{d,s} \in \{1\ldots K\}$
be the intruding topic selected by subject $s$ for document $d$ on
model $m$ and let $j^m_{d,*}$ denote the ``true'' intruder, i.e., the
one generated by the model.  We define the topic log odds as the log
ratio of the probability mass assigned to the true intruder to the
probability mass assigned to the intruder selected by the subject,
\begin{equation}\label{eq:tlo}
  \mathrm{TLO}^m_d = (\textstyle \sum_{s} \log \hat{\theta}^m_{d,j^m_{d,*}} - \log \hat{\theta}^m_{d,j^m_{d,s}}) / S.
\end{equation}
The higher the value of $\mathrm{TLO}^m_d$, the greater the
correspondence between the judgments of the model and the subjects.
The upper bound on $\mathrm{TLO}^m_d$ is 0.  This is achieved when the
subjects choose intruders with a mixture proportion no higher than the
true intruder's.


\myfig{fig:odds} shows boxplots of the topic log odds for the three
models.  As with model precision, LDA and pLSI generally outperform
CTM.  Again, this trend runs counter to CTM's superior performance on
predictive likelihood.  A histogram of the TLO of individual Wikipedia
documents is given in \myfig{fig:topic_precision} (right) for the
fifty-topic LDA model.  Documents about very specific, unambiguous
concepts, such as ``Lindy Hop,'' have high TLO because it is easy for
both humans and the model to assign the document to a particular
topic.  When documents express multiple disparate topics, human
judgments diverge from those of the model.  At the low end of the
scale is the article ``Book'' which touches on diverse areas such as
history, science, and commerce.  It is difficult for LDA to pin down
specific themes in this article which match human perceptions.

\myfig{fig:prec_vs_lhood} (bottom row) shows that, as with model
precision, increasing predictive likelihood does not imply improved
topic log odds scores.  While the topic log odds are nearly constant
across all numbers of topics for LDA and pLSI, for CTM topic log odds
and predictive likelihood are negatively correlated, yielding the
surprising conclusion that higher predictive likelihoods do not lead
to improved model interpretability.

% [1] "nyt"

% 	Spearman's rank correlation rho

% data:  both.2$x and both.2$log.likelihood 
% S = 154, p-value = 0.463
% alternative hypothesis: true rho is not equal to 0 
% sample estimates:
%        rho 
% -0.2833333 

% [1] "wikipedia10k"

% 	Spearman's rank correlation rho

% data:  both.2$x and both.2$log.likelihood 
% S = 146, p-value = 0.5809
% alternative hypothesis: true rho is not equal to 0 
% sample estimates:
%        rho 
% -0.2166667 



