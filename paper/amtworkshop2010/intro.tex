%
% topic models are widely used
%  - text words
% what topic models do
%  - take a corpus
%  - find patterns /dist on words
%  - associate docs /w themes
%  - docs exhibit multiple patterns
%
% these modeling assumptions are useful.
%  - anecdotal result
%  - no debugging
%  - model verification
%
% different assumptions lead to different models
% these models are compared across models/# topics using perplexity etc.
% decomposition is of intrinsic interest~\cite{..,..,..}
% cite hall, griffiths <- bayes factors, mimno
%
% different way of doing model selection
%

\section{Introduction}

% !!! put this in somewhere

% This versatile class of models has also been adapted to areas as
% diverse as population genetics~\cite{populationstructure}, computer
% vision~\cite{feifei:hdp}, link structure~\cite{Chang:2009uu,
%   McCallum:2005bq}, and temporal data~\cite{Gruber:2007jx}.


Probabilistic topic models have become popular tools for the
unsupervised analysis of large document collections~\cite{blei-09}.
These models posit a set of latent \emph{topics}, multinomial
distributions over words, and assume that each document can be
described as a mixture of these topics.  With algorithms for fast
approxiate posterior inference, we can use topic models to discover
both the topics and an assignment of topics to documents from a
collection of documents.  (See \myfig{fig:nyttopics:big}.)

These modeling assumptions are useful in the sense that, empirically,
they lead to good models of documents.  They also anecdotally
lead to semantically meaningful decompositions of them: topics tend to
place high probability on words that represent concepts, and documents
are represented as expressions of those concepts.  Perusing the
inferred topics is effective for model verification and for ensuring
that the model is capturing the practitioner's intuitions about the
documents.  Moreover, producing a human-interpretable decomposition of
the texts can be a goal in itself, as when browsing or summarizing a
large collection of
documents.

% dmb: cite mimnos digital libraries publication here.  possibly cite
% the stuff that comes later on when you talk about rexa, etc.

In this spirit, much of the literature comparing different topic models presents examples of topics and examples of document-topic assignments to help understand a model's mechanics.  Topics also can help users discover new content via corpus exploration~\cite{mimno-07a}.  The presentation of these topics serves, either explicitly or implicitly, as a qualitative evaluation of the latent space, but there is no explicit \emph{quantitative} evaluation of them.  Instead, researchers employ a variety of metrics of model fit, such as perplexity or held-out likelihood.  Such measures are useful for evaluating the predictive model, but do not address the more explatory goals of topic modeling.

In this paper, we present a method for measuring the
interpretatability of a topic model.  We devise two human evaluation
tasks to explicitly evaluate both the quality of the topics inferred
by the model and how well the model assigns topics to documents.  The
first, \emph{word intrusion}, measures how semantically ``cohesive''
the topics inferred by a model are and tests whether topics correspond
to natural groupings for humans.  The second, \emph{topic intrusion},
measures how well a topic model's decomposition of a document as a
mixture of topics agrees with human associations of topics with a
document.  We report the results of a large-scale human study of these
tasks, varying both modeling assumptions and number of topics.  We
show that these tasks capture aspects of topic models not measured by
existing metrics and--surprisingly--models which achieve better
predictive perplexity often have less interpretable latent spaces.
