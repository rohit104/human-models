\documentclass[11pt,letterpaper]{article}
\usepackage{naaclhlt2010}
\usepackage{times}
\usepackage{latexsym}
\usepackage{amsmath,amsthm, amssymb}
\usepackage{multirow}
\usepackage{microtype}
\usepackage{subfigure}
\usepackage{bm}
\usepackage{graphicx}
\usepackage{graphics}
\usepackage{bm}
\usepackage{multirow}
\usepackage{dsfont}

%%%% EVIL %%%%%%
\usepackage[compact,small]{titlesec}
\usepackage[small]{caption}
\usepackage{mdwlist}
%\setlength{\belowcaptionskip}{0pt}
\setlength{\textfloatsep}{8pt}
%\setlength{\floatsep}{0pt}
%%%% EVIL %%%%%%

\newcommand{\myfig}[1]{Figure~\ref{#1}}
\newcommand{\myeq}[1]{Equation~\ref{#1}}
\newcommand{\mytab}[1]{Table~\ref{#1}}
\newcommand{\wordset}[1]{\texttt{\{#1\}}}
\newcommand{\word}[1]{\texttt{#1}}
\newcommand{\mysec}[1]{Section~\ref{#1}}

%%% Temporary
%\usepackage{fancyhdr}
%\chead{{\bf DRAFT COPY: DO NOT CITE OR DISTRIBUTE}}
%\lhead{}
%\rhead{}
%\pagestyle{fancy}

\title{Human sampling of topic models}

\author{}

\begin{document}

%% \makeanontitle
\maketitle
\vspace{-.1in}
\begin{abstract}%
  Probabilistic topic models are a popular tool for the unsupervised
  analysis of text, providing both a predictive model of future text
  and a latent topic representation of the corpus.  Practitioners
  typically assume that the latent space is semantically meaningful.
  It is used to check models, summarize the corpus, and guide
  exploration of its contents.  However, whether the latent space is
  interpretable is in need of quantitative evaluation.  In this paper,
  we present new quantitative methods for measuring semantic meaning
  in inferred topics.  We back these measures with large-scale user
  studies, showing that they capture aspects of the model that are
  undetected by previous measures of model quality based on held-out
  likelihood.  Surprisingly, topic models which perform better on
  held-out likelihood may infer less semantically meaningful topics.
\end{abstract}

\section{Introduction}
Probabilistic topic models have become popular tools for the
unsupervised analysis of large document
collections~\cite{deerwester-90,griffiths02probabilistic,blei-09}.
These models posit a set of latent \emph{topics}, multinomial
distributions over words, and assume that each document can be
described as a mixture of these topics.  With algorithms for fast
approximate posterior inference, we can use topic models to discover
both the topics and an assignment of topics to documents from a
collection of documents.  (See \myfig{fig:nyttopics:big}.)

These modeling assumptions are useful in the sense that, empirically,
they lead to good models of documents~\cite{wallach-09}.  However,
recent work has explored how these assumptions correspond to humans'
understanding of language~\cite{Chang:2009fk,griffiths-06,mei-07}.  Focusing
on the latent space, i.e., the inferred mappings between topics and
words and between documents and topics, this work has discovered that
although there are some suggestive correspondences between human
semantics and topic models, they are often discordant.

In this paper we build on this work to further explore how humans
relate to topic models.  But whereas previous work has focused on
the results of topic models, here we focus on the process by which
these models are learned.  Topic models lend themselves to sequential
procedures through which the latent space is inferred; these
procedures are in effect programmatic encodings of the modeling
assumptions.  By substituting key steps in this program with human
judgments, we obtain insights into the semantic model conceived by
humans.

Here we present a novel task, \emph{tag-and-cluster}, which asks
subjects to simultaneously annotate a document and cluster that
annotation.  This task simulates the sampling step of the collapsed
Gibbs sampler (described in the next section), except that the
posterior defined by the model has been replaced by human judgments.
The task is quick to complete and is robust against noise.  We report
the results of a large-scale human study of this task, and show that
humans are indeed able to construct a topic model in this fashion, and
that the learned topic model has semantic properties distinct from
existing topic models.  We also demonstrate that the judgments can be
used to guide computer-learned topic models towards models which are
more concordant with human intuitions.

\section{Topic models and their evaluations}
\label{sec:models}
%% \subsection{Topic models in a nutshell}

% Topic models discover patterns of word usage in a corpus.  Related
% words often appear in similar documents, and topic models can discover
% clusters of words that share context.  Because words in these clusters
% often seem to ``make sense'' together, they are described as topics.
% For illustration, three topics from a topic model called latent
% Dirichlet allocation (LDA)~\cite{blei-03} are shown in
% Figure~\ref{fig:nyttopics:topic}.

% Topic models are unsupervised; they do not require labels or
% annotations.  Instead, they discover latent topics directly from the
% data.  In addition to finding topics, topic models also assign
% mixtures of these topics to documents.  This is illustrated in
% Figure~\ref{fig:nyttopics:doc}. 

% We restrict ourselves to exchangeable topic models, in which the order
% of the words within a document does not matter.  The models we
% consider are also probabilistic, in the sense that they seek to
% approximate the vector of observed word counts as a mixture of topic
% distributions; fitting one of these topic models entails finding a
% point on the topic simplex that best describes the document.

Topic models posit that each document is expressed as a mixture of
topics.  These topic proportions are drawn once per document, and the
topics are shared across the corpus.  In this paper we will consider
topic models that make different assumptions about the topic
proportions.  Probabilistic Latent Semantic Indexing
(pLSI)~\cite{hofmann-99} makes no assumptions about the document topic
distribution, treating it as a distinct parameter for each document.
Latent Dirichlet allocation (LDA)~\cite{blei-03} and the correlated
topic model (CTM)~\cite{blei-06} treat each document's topic
assignment as a multinomial random variable drawn from a symmetric
Dirichlet and logistic normal prior, respectively.

While the models make different assumptions, inference algorithms for all of
these topic models build the same type of latent space: a collection of topics
for the corpus and a collection of topic proportions for each of its
documents.  While this common latent space has explored for over two decades,
its interpretability remains unmeasured.

\begin{figure}[t]
\centering
%  \hspace{-5cm}
\subfigure[Topics]{
  \includegraphics[scale=0.35]{figures/nyt_topics.pdf}
  \label{fig:nyttopics:topic}
}
    \hspace{0.4in}
    \subfigure[Document Assignments to Topics]{
      \includegraphics[scale=0.35]{figures/nyt_documents.pdf}
      \label{fig:nyttopics:doc}
    }
    \caption{The latent space of a topic model consists of topics,
      which are distributions over words, and a distribution over these
      topics for each document.  On the left are three topics from a
      fifty topic LDA model trained on articles from the New York
      Times.  On the right is a simplex depicting the distribution
      over topics associated with seven documents.  The line from each
      document's title shows the document's position in the topic
      space.}
\label{fig:nyttopics:big}
\end{figure}


% cw: i mask the following paragraph, it seems duplicated with the next subsection

%Because the field of topic models is so vibrant, we cannot mention
%every variant or application.  Considered broadly, topic models
%encompass everything from a mixture of unigrams~\cite{Nigam-00} to
%hierarchical non-parametric methods~\cite{blei-07} and have found
%applications in collaborative filtering~\cite{Marlin:2004}, computer
%vision~\cite{feifei:hdp}, and genetics~\cite{populationstructure}.

% jbg: More detail about LDA / CTM?  I'll leave that until later, I guess.

% jbg: I had originally planned to put in DTM, SLDA, etc. here, but
% now I'm thinking that would distract from the flow.  Perhaps that
% best belongs in the intro?


% An introduction to topic models must begin with latent semantic
% analysis , which came from the psychology
% community.  Although we focus on later probabilistic topic models, LSA
% laid the foundation for those models and set important precedents for
% evaluation, so we briefly mention its assumptions.  LSA represents a
% corpus as a matrix $X$ where $X_{i,j}$ is the number of times word $j$
% appears in document $j$.  This matrix is then approximated using
% singular value decomposition.

\subsection*{Pay no attention to the latent space behind the model}

Although we focus on probabilistic topic models, the field began in
earnest with latent semantic analysis (LSA)~\cite{landauer-97}.  LSA,
the basis of pLSI's probabilistic formulation, uses linear algebra to
decompose a corpus into its constituent themes.  Because LSA
originated in the psychology community, early evaluations focused on
replicating human performance or judgments using LSA: matching
performance on standardized tests, comparing sense distinctions, and
matching intuitions about synonymy (these results are reviewed
in~\cite{landauer-02}).  In information retrieval, where LSA is known
as latent semantic indexing (LSI)~\cite{deerwester-90}, it is able to
match queries to documents, match experts to areas of expertise, and
even generalize across languages given a parallel
corpus~\cite{berry-95}.

The reticence to look under the hood of these models has persisted
even as models have moved from psychology into computer science with
the development of pLSI and LDA.  Models either use measures based on
held-out likelihood~\cite{blei-03,blei-06} or an external task that is
independent of the topic space such as sentiment
detection~\cite{titov-08} or information retrieval~\cite{wei-06}.
This is true even for models engineered to have semantically coherent
topics~\cite{boyd-graber-07}.

For models that use held-out likelihood, Wallach et
al.~\cite{wallach-09} provide a summary of evaluation
techniques. These metrics borrow tools from the language modeling
community to measure how well the information learned from a corpus
applies to unseen documents.  These metrics generalize easily and
allow for likelihood-based comparisons of different models or
selection of model parameters such as the number of topics.  However,
this adaptability comes at a cost: these methods only measure the
probability of observations; the internal representation of the models
is ignored.

Griffiths et al.~\cite{griffiths-06} is an important exception to the
trend of using external tasks or held-out likelihood.  They showed
that the number of topics a word appears in correlates with how many
distinct senses it has and reproduced many of the metrics used in the
psychological community based on human performance.  However, this is
still not a deep analysis of the structure of the latent space, as it
does not examine the structure of the topics themselves.

We emphasize that not measuring the internal representation of topic
models is at odds with their presentation and development.  Most topic
modeling papers display qualitative assessments of the inferred topics
or simply assert that topics are semantically meaningful, and
practitioners use topics for model checking during the development
process.  Hall et al.~\cite{hall-08}, for example, used latent topics
deemed historically relevant to explore themes in the scientific
literature.  Even in production environments, topics are presented as
themes: Rexa (http://rexa.info), a scholarly publication search
engine, displays the topics associated with documents.  This implicit
notion that topics have semantic meaning for users has even
motivated work that attempts to automatically label
topics~\cite{mei-07}.  Our goal is to measure the success of
interpreting topic models across number of topics and modeling
assumptions.


% jbg: Does JSTOR have something we can cite yet?

% dmb: above, i think that rexa etc should be cited earlier. (see the
% dmb in the intro.)  i wonder if a variant of that last paragraph can
% appear in the introduction and then merely be reiterated here.  this
% is on p3, which isn't so bad, but p1 is better.

%\subsection{Examining Topic Coherence and Document Relevance}
%\label{sec:coherence}

% jbg: Questions we should answer in this question but which haven't been yet:
%
% - Can users choose not to answer a question?  If so, how is it
% scored?  If not, how are they prevented from skipping? 

% - Are there checks to ensure folks don't click randomly?  (I assume
% we're only using folks with good feedback, but often there are more
% checks built in to MT experiments ...  we don't right now (right?),
% although "reveal answer" might make people thing we do.)  

% - For the "answers don't bias" claim, do we have stronger data to back
% this up?

\section{Using human judgments to examine the topics}
\label{sec:tasks}
% Put little examples of tasks in figures; for word intrusion, put up an
% example of ``good'' topic and one example of ``bad'' topic to build
% intuition about why this task is meaningful.

% 1. The main goals of the tasks

Although there appears to be a longstanding assumption that the latent
space discovered by topic models is meaningful and useful, evaluating
such assumptions is difficult because discovering topics is an
unsupervised process.  There is no gold-standard list of topics to
compare against for every corpus.  Thus, evaluating the latent space
of topic models requires us to gather exogenous data.

In this section we propose two tasks that create a formal setting
where humans can evaluate the two components of the latent space of a
topic model.  The first component is the makeup of the topics.  We
develop a task to evaluate whether a topic has human-identifiable
semantic coherence.  This task is called \emph{word intrusion}, as
subjects must identify a spurious word inserted into a topic.  The
second task tests whether the association between a document and a
topic makes sense.  We call this task \emph{topic intrusion}, as the
subject must identify a topic that was not associated with the
document by the model.

% Our experiments are designed for people using Amazon's Mechanical
% Turk~\footnote{http://www.mturk.com} system, an online clearinghouse
% for tasks which require human-level intelligence such as developing
% gold-standard data for natural language processing~\cite{snow-08} or
% labeling images~\cite{imagenet-cvpr09}.  People use Mechanical Turk to
% perform short tasks for a small fee; we will use the term ``subjects''
% to refer to individuals who chose to complete our task.

% 2. Designed two tasks
\subsection{Word intrusion}
\label{sec:wordintrusion}
To measure the coherence of these topics, we develop the \emph{word
intrusion} task; this task involves evaluating the latent space
presented in Figure~\ref{fig:nyttopics:topic}.  In the word intrusion
task, the subject is presented with six randomly ordered words.  The
task of the user is to find the word which is out of place or does not
belong with the others, i.e., the \emph{intruder}.
Figure~\ref{fig:intruding_word} shows how this task is presented to
users.

\begin{figure*}[t]
\centering

\includegraphics[width=0.90\textwidth]{figures/screenshots.png}

\caption{Screenshots of our two human tasks. In the word intrusion
  task (left), subjects are presented with a set of words and asked to
  select the word which does not belong with the others.  In the
  \emph{topic intrusion} task (right), users are given a document's
  title and the first few sentences of the document.  The users must
  select which of the four groups of words does not belong.}
\label{fig:intruding_word}
\end{figure*}

When the set of words minus the intruder makes sense together, then
the subject should easily identify the intruder.  For example, most
people readily identify \word{apple} as the intruding word in the set
\wordset{dog, cat, horse, apple, pig, cow} because the remaining
words, \wordset{dog, cat, horse, pig, cow} make sense together ---
they are all animals.  For the set \wordset{car, teacher, platypus,
  agile, blue, Zaire}, which lacks such coherence, identifying
the intruder is difficult.  People will typically choose an
intruder at random, implying a topic with poor coherence.

% sgerrish: I removed this paragraph, at least until we can measure this.
% The inter-subject agreement on the word intrusion task yields a
% measure of how semantically meaningful a set of words is: when all of
% the users agree on the intruder the set is semantically coherent; when
% none of them agree then the set is not.  

% jbg: I removed this footnote
% \footnote{The top words can be ranked either according to their
%   absolute probability mass in that topic, or according to their
%   probability mass in that topic relative to their unigram
%   probabilities.}

In order to construct a set to present to the subject, we first select
at random a topic from the model.  We then select the five most probable
words from that topic.  In addition to these words, an intruder word
is selected at random from a pool of words with low probability in the
current topic (to reduce the possibility that the intruder comes from
the same semantic group) but high probability in some other topic (to
ensure that the intruder is not rejected outright due solely to
rarity).  All six words are then shuffled and presented to the subject.

% In addition to being able to measure the semantic coherence of the
% topics using inter-subject agreement as described above, this task
% also enables the evaluation of whether the semantic divisions posited
% by a topic model correspond to those used by humans.  If the intruder
% as predicted by the topic model matched what the subjects thought was
% the intruder, then the topic found by the topic model resembles a set
% that our subjects believe is semantically coherent.

% While this
% approach also requires that models are able to select coherent topics,
% the first task provides an indication of whether the provided topics
% are reasonable.

For both the word intrusion and topic intrusion tasks, subjects were
instructed to focus on the meanings of words, not their syntactic
usage or orthography.  We also presented subjects with the option of
viewing the ``correct'' answer after they submitted their own
response, to make the tasks more engaging.  Here the ``correct''
answer was determined by the model which generated the data, presented
as if it were the response of another user.  At the same time,
subjects were encouraged to base their responses on their own
opinions, not to try to match other subjects' (the models')
selections.  In small experiments, we have found that this extra
information did not bias subjects' responses.


\section{Experimental results}
\label{sec:experiments}

For our experiments with Amazon Mechanical
Turk~\footnote{http://www.mturk.com.}, we prepare two randomly-chosen,
100-document subsets of English
Wikipedia~\footnote{http://en.wikipedia.org}.  For convenience, we
denote these two sets of documents as \emph{set1} and \emph{set2}.
For each document, we keep only the first 150 words for our
experiments.  Because of the encyclopedic nature of the corpus, the
first 150 words typically provides a broad overview of the themes in
the article.

To
prepare data for human subjects to review, we fit three different
topic models on two corpora.  In this section, we describe how we
prepared the corpora, fit the models, and created the tasks described
in \mysec{sec:tasks}.  We then present the results of these human
trials and compare them to metrics traditionally used to evaluate
topic models.

In this work we study three topic models: probabilistic latent
semantic indexing (pLSI)~\cite{hofmann-99}, latent Dirichlet
allocation (LDA)~\cite{blei-03}, and the correlated topic model
(CTM)~\cite{blei-06}, which are all mixed membership
models~\cite{Erosheva:2004ph}.  The number of latent topics, $K$, is a free
parameter in each of the models; here we explore this with $K=50$,
$100$ and $150$.  The remaining parameters -- $\beta_k$, the topic
multinomial distribution for topic $k$; and $\theta_d$, the topic
mixture proportions for document $d$ -- are inferred from data.  The
three models differ in how these latent parameters are inferred.

\begin{list}{\textbf}{\leftmargin=0pt}
\setlength{\itemindent}{25pt}
\setlength{\itemsep}{0pt}
\item[{\bf pLSI}] In pLSI, the topic mixture proportions $\theta_d$ are a parameter for
% Conditioned on $\theta_d$, the likelihood of the observed bag of words
% for document $d$, $\bm{w}_d$, for all three models exhibits the same
% probabilistic form,
% \begin{align}\label{eq:likelihood}
% p(\bm{w}_d|\theta_d) = \prod_{n=1}^{N}\sum_{z_{n}=1}^K p(w_{n}|z_{n}, \beta_{1:K}) p(z_{n}|\theta).
% \end{align}
% However, the different assumptions about the mixture proportions
% $\theta_d$ used by the three models lead to significantly different
% latent spaces.
  each document.  Thus, pLSI is not a fully generative model, and the number
  of parameters grows linearly with the number of documents.
% jc1: ???
% Moreover due to its non-generative nature, inference needs to be
% done in the ``folding-in'' process~\cite{hofmann-99}.
  We fit pLSI using the EM algorithm~\cite{Dempster:1977} but
  regularize pLSI's estimates of $\theta_d$ using pseudo-count
  smoothing, $\alpha = 1$.  

% Under these settings, the maximum \emph{a
%     posteriori} (MAP) estimates found by pLSI correspond to those
%   found by LDA with a symmetric Dirichlet prior on $\theta_d$ with
%   parameter $\alpha$~\cite{Girolami-03}.

\item[{\bf LDA}] LDA is a fully generative model of documents where
  the mixture proportions $\theta_d$ are treated as a random variable
  drawn from a Dirichlet prior
  distribution. %that is%$\theta \sim \textrm{Dirichlet}(\alpha)$.
% \begin{align}\label{eq:dirichlet}
% p(\theta | \alpha) = \frac{\Gamma(\sum_{k=1}^K\alpha_k)}{\prod_{k=1}^K \Gamma(\alpha_k) } \prod_{k=1}^K\theta_k^{\alpha_k-1},
% \end{align}
% where $\Gamma(\cdot)$ is the Gamma function.  
% Due to its generative nature, inference on unseen documents is
% well-defined.  
  Because the direct computation of the posterior is intractable, we
  employ variational inference~\cite{blei-03} and set the symmetric
  Dirichlet prior parameter, $\alpha$, to 1.

\item[{\bf CTM}] In LDA, the components of $\theta_d$ are nearly independent
  (i.e., $\theta_d$ is statistically neutral).
% This leads to the possibly-unrealistic modeling assumption
% that the presence of one topic in a document is not correlated with
% the presence of another topic in the same document. 
CTM allows for a richer covariance structure between topic proportions
by using a logistic normal prior over the topic mixture proportions
$\theta_d$.  For each topic, $k$, a real $\gamma$ is drawn from a
normal distribution and exponentiated.  This set of $K$ non-negative
numbers are then normalized to yield $\theta_d$.  Here, we train the
CTM using variational inference~\cite{blei-06}.
\end{list}

% Using this formulation, CTM is able to capture the correlations among
% different topics and provides a natural way of visualizing the topics
% using the correlations.


%% {\bf TODO: Chong, it seems not fair not to optimize $\alpha$ for pLSI and LDA but optimize that for CTM?}

We train each model on two corpora.  For each corpus, we apply a part
of speech tagger~\cite{schmid-94} and remove all tokens tagged as
proper nouns (this was for the benefit of the human subjects; success
in early experiments required too much encyclopedic knowledge).  Stop
words~\cite{loper-02} and terms occurring in fewer than five documents
are also removed. The two corpora we use are 1.) a collection of 8447
articles from the \textit{New York Times} from the years 1987 to 2007
with a vocabulary size of 8269 unique types and around one million
tokens and 2.)  a sample of 10000 articles from \textit{Wikipedia}
(http://www.wikipedia.org) with a vocabulary size of 15273 unique
types and three million tokens.
% \item \textbf{IMDB}: a collection of xxx reviews from the
%   \textit{Internet Movie Database}
%   (IMDb)\footnote{http://www.imdb.com/}, with a vocabulary size xxx
%   and around xxx words.

\subsection{Evaluation using conventional objective measures}
\label{sec:evalmetrics}

There are several metrics commonly used to evaluate topic models in
the literature~\cite{wallach-09}.  Many of these metrics are
\emph{predictive} metrics; that is, they capture the model's ability
to predict a \emph{test set} of unseen documents after having learned
its parameters from a \emph{training set}.  In this work, we set aside
20\% of the documents in each corpus as a test set and train on the
remaining 80\% of documents.  We then compute predictive rank and
predictive log likelihood.

To ensure consistency of evaluation across different models, we follow
Teh et al.'s~\cite{TehKurWel2008} approximation of the predictive
likelihood $p(\textbf{w}_d|D_{\textrm{train}})$ using $p({\bm
  w}_d|D_{\textrm{train}}) \approx p({\bm w}_d|\hat{\theta}_d)$, where
$\hat{\theta}_d$ is a point estimate of the posterior topic
proportions for document $d$. For pLSI $\hat{\theta}_d$ is the MAP
estimate; for LDA and CTM $\hat{\theta}_d$ is the mean of the
variational posterior.
With this information, we can ask what words the model believes will
be in the document and compare it with the document's actual
composition.  Given document $\bm w_d$, we first estimate
$\hat{\theta}_d$ and then for every word in the vocabulary, we compute
$p(w|\hat{\theta}_d) = \sum_z p(w|z)p(z|\hat{\theta}_d)$.  Then we
compute the average rank for the terms that actually appeared in document
$\bm w_d$ (we follow the convention that lower rank is better).


%\paragraph{Predictive perplexity}

% The predictive perplexity of a heldout set $D_{heldout}$ given the
% training set $D_{train}$ is defined as
% \begin{align*} \label{eq:perplexity}
%   {\textrm{perplexity}} = \exp\left\{-\frac{\sum_{d\in D_{\textrm{heldout}}}\log
%       p(\textbf{w}_d|D_{\textrm{train}})}{\sum_{d\in D_{heldout}}N_d}\right\}.
% \end{align*}


% \begin{table}
% \caption{Perplexity comparison for different models on three corpora.}
% \label{fig:perplexity}
% %\vskip 0.15in
% \begin{center}
% \begin{small}
% \begin{sc}
% \begin{tabular}{c|c|cc}
%   \hline
% \multicolumn{4}{c}{Per-word perplexity} \\
% \hline
% \#Topics & Model & NYT & Wikipedia \\
% \hline
% \multirow{3}{*}{50}
% & LDA & 1641.82 & 1980.65 \\
% & CTM & 1755.38 & 2517.14 \\
% & pLSI &  &  \\
% \hline
% \multirow{3}{*}{100}
% & LDA & 1542.21 & 1836.51 \\
% & CTM & 1652.53 & 2327.30 \\
% & pLSI &  &  \\
% \hline
% \multirow{3}{*}{150}
% & LDA & 1485.04 & 1755.31 \\
% & CTM &         & 2227.58 \\
% & pLSI &   \\
% \hline
% \end{tabular}
% \end{sc}
% \end{small}
% \end{center}
% \end{table}

%% We need to decide if we want to keep these metrics...
% \paragraph{Word predictions}

% The first metric is predictive perplexity, which is an objective
% metric used extensively for comparing topic models in the
% literature.  Perplexity is a measure of the
% information content of unseen documents that cannot be captured by the
% model.  Lower perplexity means that the model is better able to
% explain the language of test documents.


The average word likelihood and average rank across all documents in
our test set are shown in Table~\ref{tab:word-prediction}.  These
results are consistent with the values reported in the
literature~\cite{blei-03,blei-06}; in most cases CTM performs best,
followed by LDA.

\begin{table*}
  \caption{Two predictive metrics: predictive log likelihood/predictive rank.  Consistent with values reported in the literature, CTM generally performs the best, followed by LDA, then pLSI.  The bold numbers indicate the best performance in each row.}
\label{tab:word-prediction}
\centering
\footnotesize
\begin{tabular}{rlllll}
  topic & & & & & \\
  \hline
  0 & railway & lighthouse & rail & huddersfield & station \\ 
  1 & school & college & education & history & conference \\ 
  2 & catholic & church & film & music & actor \\ 
  3 & runners & team & championships & match & racing \\ 
  4 & engine & company & power & dwight & engines \\ 
  5 & university & london & british & college & county \\ 
  6 & food & novel & book & series & superman \\ 
  7 & november & february & april & august & december \\ 
  8 & paint & photographs & american & austin & black \\ 
  9 & war & history & army & american & battle \\ 
   \hline
\end{tabular}
\vspace{0.2in}
\end{table*}

\subsection{Analyzing human evaluations}

The tasks described in \mysec{sec:tasks} were offered on Amazon
Mechanical Turk (http://www.mturk.com), which allows workers (our pool
of prospective subjects) to perform small jobs for a fee through a Web
interface.  No specialized training or knowledge is typically expected
of the workers.  Amazon Mechanical Turk has been successfully used in
the past to develop gold-standard data for natural language
processing~\cite{snow-08} and to label images~\cite{imagenet-cvpr09}.
For both the word intrusion and topic intrusion tasks, we presented
each worker with jobs containing ten of the tasks described in
\mysec{sec:tasks}.  Each job was performed by 8 separate workers, and
workers were paid between \$0.07 -- \$0.15 per job.

\paragraph{Word intrusion}
As described in \mysec{sec:wordintrusion}, the word intrusion task
measures how well the inferred topics match human concepts (using
\emph{model precision}, i.e., how well the intruders detected by the
subjects correspond to those injected into ones found by the topic model).  

Let $\omega^{m}_{k}$ be the index of the intruding word among the words generated
from the $k^{th}$ topic inferred by model $m$.  Further let $i^m_{k, s}$ be
the intruder selected by subject $s$ on the set of words generated from the
$k$th topic inferred by model $m$ and let $S$ denote the number of subjects.  We
define model precision by the fraction of subjects agreeing with the model,
\begin{equation}
  \mathrm{MP}^m_k = \textstyle \sum_{s} \mathds{1}(i^m_{k,s} = \omega^{m}_{k}) / S.
  \label{eq:mp}
\end{equation}


\myfig{fig:precision} shows boxplots of the precision for the three
models on the two corpora.  In most cases LDA performs best. Although
CTM gives better predictive results on held-out likelihood, it does
not perform as well on human evaluations. This may be because CTM
finds correlations between topics and correlations within topics are
confounding factors; the intruder for one topic might be selected from
another highly correlated topic.  The performance of pLSI degrades
with larger numbers of topics, suggesting that
overfitting~\cite{blei-03} might affect interpretability as well as
predictive power.

\myfig{fig:topic_precision} (left) shows examples of topics with high
and low model precisions from the NY Times data fit with LDA using 50
topics. In the example with high precision, the topic words all
coherently express a painting theme.  For the low precision example,
 ``taxis'' did not fit in with the other
political words in the topic, as $87.5\%$ of subjects chose ``taxis''
as the intruder.


The relationship between model precision, $\mathrm{MP}^m_k$, and the
model's estimate of the likelihood of the intruding word in
\myfig{fig:prec_vs_lhood} (top row) is surprising.  The highest
probability did not have the best interpretability; in fact, the trend
was the opposite.  This suggests that as topics become more
fine-grained in models with larger number of topics, they are less
useful for humans.  The downward sloping trend lines in
\myfig{fig:prec_vs_lhood} implying that the models are often trading
improved likelihood for lower interpretability.


The model precision showed a negative correlation (Spearman's $\rho =
-0.235$ averaged across all models, corpora, and topics) with the
number of senses in WordNet of the words displayed to the
subjects~\cite{Miller90} and a slight positive correlation ($\rho =
0.109$) with the average pairwise Jiang-Conrath similarity of
words\footnote{Words without entries in WordNet were ignored; polysemy
  was handled by taking the maximum over all senses of words.  To
  handle words in the same synset (e.g. ``fight'' and ``battle''), the
  similarity function was capped at 10.0.}~\cite{jiang-97}.  


\section{Discussion}

We presented the first validation of the assumed coherence and
relevance of topic models using human experiments.
% This validation is only possible today because of the advent of new
% services like Amazon Mechanical Turk.
For three topic models, we demonstrated that traditional metrics do
not capture whether topics are coherent or not.  Traditional metrics
are, indeed, negatively correlated with the measures of topic quality
developed in this paper.  Our measures enable new forms of model
selection and suggest that practitioners developing topic models should
thus focus on evaluations that depend on real-world task performance
rather than optimizing likelihood-based measures.

In a more qualitative vein, this work validates the use of topics for
corpus exploration and information retrieval.  Humans appreciate the
semantic coherence of topics and can associate the same documents with
a topic that a topic model does.  An intriguing possibility is the
development of models that explicitly seek to optimize the measures we
develop here either by incorporating human judgments into the
model-learning framework or creating a computational proxy that
simulates human judgments.

\bibliographystyle{naaclhlt2010}

\small

\bibliography{journal-abbrv,nlp}

\end{document}
